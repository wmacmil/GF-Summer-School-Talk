\documentclass{beamer}
\usepackage[T1]{fontenc}
\usepackage[utf8]{inputenc}
\usepackage{lmodern}
\usepackage[english]{babel}

\usepackage{geometry}
\usepackage{setspace}

\usepackage{latex/agda}
\usepackage{unicode-math}
\setmathfont{XITS Math}

\setmainfont{DejaVu Serif}
\setsansfont{DejaVu Sans}
\setmonofont{DejaVu Sans Mono}


\usepackage{newunicodechar}
\newunicodechar{→}{\ensuremath{\mathnormal\to}}
\newunicodechar{ℕ}{\ensuremath{\mathbb{N}}}


\usepackage{xcolor}
\usepackage[normalem]{ulem}
\usepackage{soul} 
\usepackage{amsmath} 
\usepackage{amssymb}

% \usepackage{enumitem} %personalizza gli elenchi
% \usepackage{amsthm} %teoremi e definizioni

\usepackage{multirow}
\usepackage{multicol}
\usepackage{caption}
\usepackage{bussproofs}

\usepackage{tikz-cd}
\usetikzlibrary{matrix}

\let\oldquote\quote
\let\endoldquote\endquote

\RenewDocumentEnvironment{quote}{om}
  {\oldquote}
  {\par\nobreak\smallskip
   \hfill(#2\IfValueT{#1}{~---~#1})\endoldquote 
   \addvspace{\bigskipamount}}

% \newenvironment<>{varblock}[2][.9\textwidth]{%
%   \setlength{\textwidth}{#1}
%   \begin{actionenv}#3%
%     \def\insertblocktitle{#2}%
%     \par%
%     \usebeamertemplate{block begin}}
%   {\par%
%     \usebeamertemplate{block end}%
%   \end{actionenv}}


% \usepackage{tikz-cd}
% % \usepackage{tikz}

\usetheme{Antibes}
\usecolortheme{beaver}
\useinnertheme{rounded}
\useoutertheme{infolines}

%\titlegraphic{\includegraphics[width=25mm]{gottingen1.png}}
\title{Modeling Formal Languages in Grammatical Framework}
\subtitle{On the Grammar of Proof}
\author{Warrick Macmillan}
\date{$7^{th}$ August 2021}


\begin{document}

\begin{frame}
  \titlepage
\end{frame}


% \begin{frame}
% \frametitle{Overview}
% \tableofcontents
% \end{frame}

\section{Overview}

\subsection{Introduction}
\begin{frame}
\frametitle{Table of Contents}

\begin{enumerate}

\item Explore abstract relationships between math, CS, Type Theory, and
  Linguistics 
\item Practical and brief intro to MLTT and Agda
\item Grammars elaborating the abstractions above
\end{enumerate}
\end{frame}


\begin{frame}[fragile]
\frametitle{Abstraction Ladders}
\begin{columns}
\begin{column}{.5\linewidth}
\begin{tikzcd}
Strings \ar[d,"Lexical\ Analysis"] \ar[dd,bend right=+90.0, swap,"Front\ End"]
&[5m]
\\ Lexemes \ar[d,"Parsing"] &[5em]
\\ ASTs \ar[d,"Type\ Checker"] &[5m]
\\ Typed\ ASTs
  \ar[dd, bend left, "Code\ Generator"] 
  \ar[dd, bend right, swap, "Interpreter"] &[5m]
\\ ...
\\ Target\ Language
\end{tikzcd}
\end{column}
\begin{column}{0.6 \linewidth}
\begin{tikzcd}
  Phonemes \arrow[d, "Morhphophonological
  \\ Anaylsis" description]
  \\ Morphemes \arrow[d, "Parse"]
  \\ \{\ Syntactic\ Representation\ \} \arrow[d, "Montague"', bend right=49]
    \arrow[d, "Ranta", bend left=49] \arrow[d, "..." description]
  \\ {\{\ STLC,\ ...\ ,\ DTT\ \}} \arrow[d, "?" description]
  \\ {\{\ Nearal Encoding\ ,\ ...\ ,\ I\ Language\ \}} \arrow[d, "?" description]
  \\ Phonemes
\end{tikzcd}
\end{column}
\end{columns}
\end{frame}

\begin{frame}[fragile]
\frametitle{Computational Trinitarianism}
\centering 
\begin{tikzcd}
                                                                            &  &  & Logic \arrow[llldddd, "Denotational\ Semantics" description] \arrow[rrrdddd, "Include\ Terms" description] &  &  &                                                                                                       \\
                                                                            &  &  &                                                                                                            &  &  &                                                                                                       \\
                                                                            &  &  &                                                                                                            &  &  &                                                                                                       \\
                                                                            &  &  &                                                                                                            &  &  &                                                                                                       \\
Math \arrow[rrruuuu, "Embedded\ in\ FOL", bend left] \arrow[rrrrrr, "ITP"'] &  &  &                                                                                                            &  &  & CS \arrow[llllll, "Denotational\ Semantics", bend left] \arrow[llluuuu, "Remove\ Terms"', bend right]
\end{tikzcd}
  

\end{frame}


\begin{frame}[fragile]
% \frametitle{Linguistic  Interpretations}
\frametitle{Interpretation Language}
\begin{alertblock}{Observation 1.1}
  We acknowledge this is only semantic interpretations in these domains.
  One may decide on syntactic, pragmatic, or other ways in which to treat
  linguistics via these fields
\end{alertblock}  
\centering 
\begin{tikzcd}
     &  &  & Logic                                                                                                                     &  &  &            \\
     &  &  &                                                                                                                           &  &  &            \\
     &  &  & Linguistics \arrow[uu, "Montague\ Semantics"'] \arrow[llldd, "Distributional\ Semantics"'] \arrow[rrrdd, "TT\ Semantics"] &  &  &            \\
     &  &  &                                                                                                                           &  &  &            \\
Math &  &  &                                                                                                                           &  &  & CS\ (MLTT)
\end{tikzcd} 

\end{frame}


\begin{frame}[fragile]
\frametitle{Trinitarian Linguistics }
\centering 
\begin{tikzcd}
                                                &  &  & Logic \arrow[dd, "Embedding"] &  &  &                               \\
                                                &  &  &                               &  &  &                               \\
                                                &  &  & Linguistics                   &  &  &                               \\
                                                &  &  &                               &  &  &                               \\
Math \arrow[rrruu, "Language\ Of\ Mathematics"] &  &  &
&  &  & CS\ (MLTT) \arrow[llluu, "Meaning\ Explanations"]
\end{tikzcd} 

\end{frame}

\begin{frame}[fragile]
\centering 
\frametitle{Trinitarian Grammars}
% https://tikzcd.yichuanshen.de/#N4Igdg9gJgpgziAXAbVABwnAlgFyxMJZAZgBoAGAXVJADcBDAGwFcYkQAZCAcywGMQAX1LpMufIRTlSAFmp0mrdgFl6OABZCRIDNjwEiANlnyGLNohABhAMoAdOwAIAFMo4AVdwEotovRKIyACZTRQsQAHEAMSF5GChueCJQADMAJwgAWyRpEBwIJDIQRnoAIxhGAAUxfUlimBScEBozJUsAJXowHHoHRy5ePj7nKwBBABEAUR9hVIzsxABGGnzCmhLyqpqAy0YGppaw9k7u3qcACQgcHGGbAFU+1Q0Z7XSspCCVgsQi1vC+ZilfhMTyxQRAA
\begin{tikzcd}
                                              &  &  & Logic \arrow[dd, "Ranta\ Logic\ (CADE)"] &  &  &                                       \\
                                              &  &  &                                          &  &  &                                       \\
                                              &  &  & GF                                       &  &  &                                       \\
                                              &  &  &                                          &  &  &                                       \\
Math \arrow[rrruu, "Ranta\ Hott\ (SU\ Math)"] &  &  &                                          &  &  & CS\ (MLTT) \arrow[llluu, "cubicalTT"]
\end{tikzcd}

\end{frame}

\begin{frame}[fragile]
\frametitle{Models of  GF}
\centering 
\begin{tikzcd}
     &  &  & Logic                                                                                                                                             &  &  &            \\
     &  &  &                                                                                                                                                   &  &  &            \\
     &  &  & GF \arrow[uu, "?"'] \arrow[llldd, "Theory\ of\ Operads"']
     \arrow[rrrdd, "Implementation\ of", bend left] \arrow[rrrdd, "Agda\ Embedding", bend right] &  &  &            \\
     &  &  &                                                                                                                                                   &  &  &            \\
Math &  &  &                                                                                                                                                   &  &  & CS\ (MLTT)
\end{tikzcd}

\end{frame}

\begin{frame}[fragile]
\frametitle{Remarks}
\centering 
\begin{itemize}
% \begin{enumerate}
\item Trinitarian doctrine is in the ``formal" space
\item Trinitarian + Linguistics is partially formal, and very underexplored
\item Introduces many philosophical concerns, perhaps a rereading of
  Wittgenstein should take place in this context
% \end{enumerate}
\end{itemize}
\end{frame}

\section{Preliminaries and Perspectives}

\subsection{MLTT}

\begin{frame}

\begin{itemize}
  \item Frege : Formal Proof, Predicate Logic
  \item Russel : Type Theory to resolve his paradox
  \item Brouwer : Constructivism
\end{itemize}

\end{frame}

\begin{frame}

\begin{quote}{Per Martin-Löf, Padua Italy, June 1980}

Mathematical logic and the relation between logic and mathematics have been
interpreted in at least three different ways:
\newline

\\
i. mathematical logic as symbolic logic, or logic using mathematical symbolism; \\
ii. mathematical logic as foundations (or philosophy) of mathematics;\\
iii. mathematical logic as logic studied by mathematical methods, as a branch of mathematics.
\newline

\\
We shall here mainly be interested in mathematical logic in the second sense.
What we shall do is also mathematical logic in the first sense, but certainly
not in the third.
\end{quote}
\end{frame}

\begin{frame}
\frametitle{Syntactic Comparisons}

\begin{columns}
  
\begin{column}{0.4 \textwidth}
\begin{block}{First Order Logic}
  \begin{itemize}
    \item $\forall$
    \item $\exists$
    \item $\supset$
    \item $\wedge$
    \item $\lor$
    \item $\neg$
    \item $\top$
    \item $\bot$
    \item $=$
  \end{itemize}
\end{block}
\end{column}

\begin{column}{0.4 \textwidth}
\begin{block}{Dependent Type Theory}
  \begin{itemize}
    \item $\Pi$
    \item $\Sigma$
    \item $\to$
    \item $\times$
    \item $+$
    \item $\neg$
    \item $\top$
    \item $\bot$
    \item $\equiv$
  \end{itemize}
\end{block}
\end{column}

\end{columns}

\end{frame}

\begin{frame}
\begin{columns}
  
\begin{column}{0.4 \textwidth}
\begin{exampleblock}{Sets}
  \begin{itemize}
    \item $\mathbb{N}$
    \item $\mathbb{N} \times \mathbb{N}$
    \item $\mathbb{N} \to \mathbb{N}$
    \item $\{x|P(x)\}$
    \item $\emptyset$
    \item $?$
    \item $\cup$
    \item $?$
  \end{itemize}
\end{exampleblock}
\end{column}

\begin{column}{0.4 \textwidth}
\begin{block}{Types}
  \begin{itemize}
    \item $Nat$
    \item $Nat \times Nat$
    \item $Nat \to Nat$
    \item $\Sigma x : \_ . P(x)$
    \item $\bot$
    \item $\top$
    \item $?$
    \item $U_1$
  \end{itemize}
\end{block}
\end{column}
\end{columns}

\begin{columns}
  
\begin{column}{0.4 \textwidth}
\begin{exampleblock}{More Sets}
  \begin{itemize}
    \item $1$
    \item $(1,0)$
  \end{itemize}
\end{exampleblock}
\end{column}

\begin{column}{0.4 \textwidth}
\begin{block}{Programs}
  \begin{itemize}
    \item $suc\ zero$
    \item $(suc\ zero, zero)$
  \end{itemize}
\end{block}
\end{column}
\end{columns}

\end{frame}

\begin{frame}

\frametitle{Judgments}

\begin{columns}

\begin{column}{0.4 \textwidth}
\begin{block}{Type Theoretic Judgments}
  \begin{itemize}
  \item $T$ is a type
  \item $T$ and $T'$ are equal types
  \item $t$ is a term of type $T$
  \item $t$ and $t'$ are equal terms of type $T$
  \end{itemize}
\end{block}
\end{column}

\begin{column}{0.4 \textwidth}
\begin{block}{Mathematical Judgments}
  \begin{itemize}
  \item $P$ is a proposition
  \item $P$ is true
  \end{itemize}
\end{block}
\end{column}
\end{columns}

\\~\\
\begin{itemize}
  \item Notice that judgmental equality is uniquely type theoretic
  \item Judgments in type theory are decidable
  \item Truth (inhabitation) is not decidable
  \item More exotic judgments are available in TT, i.e. $P$ is possible.
\end{itemize}

\end{frame}


\begin{frame}

\frametitle{Important Differences}

\begin{itemize}
\item The rules of the types make explicit that they are not equivalent to those
  of classical FOL
\item An existential assertion in type theory requires data
\item Excluded middle and double negation are not admitted in MLTT
\item To be \emph{not unhappy} is clearly of a different meaning than to be \emph{happy}.
\item This makes our approach to general translation of non-constructive mathematics \emph{impossible}  (at least such that it type-checks)

\end{itemize}
\end{frame}

\begin{frame}

\begin{itemize}
\item One doesn't define logics, type systems in mathematics (e.g. metamathemeatics)
\item Encoding things like rational and real numbers in type theory are
\item already, category theorists and set theorists are at odds, (small and
  large categories), higher categories, which skeletons of categories are canonical, etc.
  incredibly difficult
\item Additionally, intensional type theory comes with two distinct notions of
  equality, judgmental/definitional/computational and propositional equality
\end{itemize}  
\end{frame}

\begin{frame}

\frametitle{Example Donkey Anaphora}
  
Interpret the sentence ``every man who owns a donkey beats it'' in MLTT via the following judgment :

\[\Pi z : (\Sigma x : man. \; \Sigma y : donkey. \; owns(x,y)). \;
  beats(\pi_1z,\pi_1(\pi_2z))\] 

We judge $\vdash man \; {:} {\rm type}$ and $\vdash donkey \; {:}
{\rm type}$. ${\rm type}$ really denotes a universe

\end{frame}

\subsection{Agda}

\begin{frame}
\frametitle{What is Agda?}

\begin{itemize}
\item Implementation of MLTT
\item Logical Framework
\item Interactive proof development environment
\item Inductive Types, Modules, Pattern Matching, & more
\end{itemize}

\end{frame}

\begin{frame}

\frametitle{Twin Prime Conjecture in Agda}

\begin{code}[hide]%
\>[0]\AgdaKeyword{module}\AgdaSpace{}%
\AgdaModule{twin-primes}\AgdaSpace{}%
\AgdaKeyword{where}\<%
\\
%
\\[\AgdaEmptyExtraSkip]%
\>[0]\AgdaKeyword{open}\AgdaSpace{}%
\AgdaKeyword{import}\AgdaSpace{}%
\AgdaModule{Data.Nat}\AgdaSpace{}%
\AgdaKeyword{renaming}\AgdaSpace{}%
\AgdaSymbol{(}\AgdaOperator{\AgdaPrimitive{\AgdaUnderscore{}+\AgdaUnderscore{}}}\AgdaSpace{}%
\AgdaSymbol{to}\AgdaSpace{}%
\AgdaOperator{\AgdaPrimitive{\AgdaUnderscore{}∔\AgdaUnderscore{}}}\AgdaSymbol{)}\<%
\\
\>[0]\AgdaKeyword{open}\AgdaSpace{}%
\AgdaKeyword{import}\AgdaSpace{}%
\AgdaModule{Data.Product}\AgdaSpace{}%
\AgdaKeyword{using}\AgdaSpace{}%
\AgdaSymbol{(}\AgdaRecord{Σ}\AgdaSymbol{;}\AgdaSpace{}%
\AgdaOperator{\AgdaFunction{\AgdaUnderscore{}×\AgdaUnderscore{}}}\AgdaSymbol{;}\AgdaSpace{}%
\AgdaOperator{\AgdaInductiveConstructor{\AgdaUnderscore{},\AgdaUnderscore{}}}\AgdaSymbol{;}\AgdaSpace{}%
\AgdaField{proj₁}\AgdaSymbol{;}\AgdaSpace{}%
\AgdaField{proj₂}\AgdaSymbol{;}\AgdaSpace{}%
\AgdaFunction{∃}\AgdaSymbol{;}\AgdaSpace{}%
\AgdaFunction{Σ-syntax}\AgdaSymbol{;}\AgdaSpace{}%
\AgdaFunction{∃-syntax}\AgdaSymbol{)}\<%
\\
\>[0]\AgdaKeyword{open}\AgdaSpace{}%
\AgdaKeyword{import}\AgdaSpace{}%
\AgdaModule{Data.Sum}\AgdaSpace{}%
\AgdaKeyword{renaming}\AgdaSpace{}%
\AgdaSymbol{(}\AgdaOperator{\AgdaDatatype{\AgdaUnderscore{}⊎\AgdaUnderscore{}}}\AgdaSpace{}%
\AgdaSymbol{to}\AgdaSpace{}%
\AgdaOperator{\AgdaDatatype{\AgdaUnderscore{}+\AgdaUnderscore{}}}\AgdaSymbol{)}\<%
\\
\>[0]\AgdaKeyword{import}\AgdaSpace{}%
\AgdaModule{Relation.Binary.PropositionalEquality}\AgdaSpace{}%
\AgdaSymbol{as}\AgdaSpace{}%
\AgdaModule{Eq}\<%
\\
\>[0]\AgdaKeyword{open}\AgdaSpace{}%
\AgdaModule{Eq}\AgdaSpace{}%
\AgdaKeyword{using}\AgdaSpace{}%
\AgdaSymbol{(}\AgdaOperator{\AgdaDatatype{\AgdaUnderscore{}≡\AgdaUnderscore{}}}\AgdaSymbol{;}\AgdaSpace{}%
\AgdaInductiveConstructor{refl}\AgdaSymbol{;}\AgdaSpace{}%
\AgdaFunction{trans}\AgdaSymbol{;}\AgdaSpace{}%
\AgdaFunction{sym}\AgdaSymbol{;}\AgdaSpace{}%
\AgdaFunction{cong}\AgdaSymbol{;}\AgdaSpace{}%
\AgdaFunction{cong-app}\AgdaSymbol{;}\AgdaSpace{}%
\AgdaFunction{subst}\AgdaSymbol{)}\<%
\\
\>[0]\AgdaKeyword{open}\AgdaSpace{}%
\AgdaModule{Eq.≡-Reasoning}\AgdaSpace{}%
\AgdaKeyword{using}\AgdaSpace{}%
\AgdaSymbol{(}\AgdaOperator{\AgdaFunction{begin\AgdaUnderscore{}}}\AgdaSymbol{;}\AgdaSpace{}%
\AgdaOperator{\AgdaFunction{\AgdaUnderscore{}≡⟨⟩\AgdaUnderscore{}}}\AgdaSymbol{;}\AgdaSpace{}%
\AgdaFunction{step-≡}\AgdaSymbol{;}\AgdaSpace{}%
\AgdaOperator{\AgdaFunction{\AgdaUnderscore{}∎}}\AgdaSymbol{)}\<%
\\
\>[0]\<%
\end{code}

\begin{code}%
\>[0]\<%
\\
\>[0]\AgdaFunction{is-prime}\AgdaSpace{}%
\AgdaSymbol{:}\AgdaSpace{}%
\AgdaDatatype{ℕ}\AgdaSpace{}%
\AgdaSymbol{→}\AgdaSpace{}%
\AgdaPrimitive{Set}\<%
\\
\>[0]\AgdaFunction{is-prime}\AgdaSpace{}%
\AgdaBound{n}\AgdaSpace{}%
\AgdaSymbol{=}\<%
\\
\>[0][@{}l@{\AgdaIndent{0}}]%
\>[2]\AgdaSymbol{(}\AgdaBound{n}\AgdaSpace{}%
\AgdaOperator{\AgdaFunction{≥}}\AgdaSpace{}%
\AgdaNumber{2}\AgdaSymbol{)}\AgdaSpace{}%
\AgdaOperator{\AgdaFunction{×}}\<%
\\
%
\>[2]\AgdaSymbol{((}\AgdaBound{x}\AgdaSpace{}%
\AgdaBound{y}\AgdaSpace{}%
\AgdaSymbol{:}\AgdaSpace{}%
\AgdaDatatype{ℕ}\AgdaSymbol{)}\AgdaSpace{}%
\AgdaSymbol{→}\AgdaSpace{}%
\AgdaBound{x}\AgdaSpace{}%
\AgdaOperator{\AgdaPrimitive{*}}\AgdaSpace{}%
\AgdaBound{y}\AgdaSpace{}%
\AgdaOperator{\AgdaDatatype{≡}}\AgdaSpace{}%
\AgdaBound{n}\AgdaSpace{}%
\AgdaSymbol{→}\AgdaSpace{}%
\AgdaSymbol{(}\AgdaBound{x}\AgdaSpace{}%
\AgdaOperator{\AgdaDatatype{≡}}\AgdaSpace{}%
\AgdaNumber{1}\AgdaSymbol{)}\AgdaSpace{}%
\AgdaOperator{\AgdaDatatype{+}}\AgdaSpace{}%
\AgdaSymbol{(}\AgdaBound{x}\AgdaSpace{}%
\AgdaOperator{\AgdaDatatype{≡}}\AgdaSpace{}%
\AgdaBound{n}\AgdaSymbol{))}\<%
\\
%
\\[\AgdaEmptyExtraSkip]%
\>[0]\AgdaFunction{twin-prime-conjecture}\AgdaSpace{}%
\AgdaSymbol{:}\AgdaSpace{}%
\AgdaPrimitive{Set}\<%
\\
\>[0]\AgdaFunction{twin-prime-conjecture}\AgdaSpace{}%
\AgdaSymbol{=}\AgdaSpace{}%
\AgdaSymbol{(}\AgdaBound{n}\AgdaSpace{}%
\AgdaSymbol{:}\AgdaSpace{}%
\AgdaDatatype{ℕ}\AgdaSymbol{)}\AgdaSpace{}%
\AgdaSymbol{→}\AgdaSpace{}%
\AgdaFunction{Σ[}\AgdaSpace{}%
\AgdaBound{p}\AgdaSpace{}%
\AgdaFunction{∈}\AgdaSpace{}%
\AgdaDatatype{ℕ}\AgdaSpace{}%
\AgdaFunction{]}\AgdaSpace{}%
\AgdaSymbol{(}\AgdaBound{p}\AgdaSpace{}%
\AgdaOperator{\AgdaFunction{≥}}\AgdaSpace{}%
\AgdaBound{n}\AgdaSymbol{)}\<%
\\
\>[0][@{}l@{\AgdaIndent{0}}]%
\>[2]\AgdaOperator{\AgdaFunction{×}}\AgdaSpace{}%
\AgdaFunction{is-prime}\AgdaSpace{}%
\AgdaBound{p}\<%
\\
%
\>[2]\AgdaOperator{\AgdaFunction{×}}\AgdaSpace{}%
\AgdaFunction{is-prime}\AgdaSpace{}%
\AgdaSymbol{(}\AgdaBound{p}\AgdaSpace{}%
\AgdaOperator{\AgdaPrimitive{∔}}\AgdaSpace{}%
\AgdaNumber{2}\AgdaSymbol{)}\<%
\\
\>[0]\<%
\end{code}


\end{frame}

\begin{frame}
\begin{block}{Mathematical Declarations}
  \begin{itemize}
    \item Theorem
    \item Proof
    \item Lemma
    \item Axiom
    \item Definition
    \item Example 
  \end{itemize}
\end{block}
\end{frame}

\begin{frame}
Boolean example

(type theoretic syntax)

\end{frame}

\subsection{Philosophical Considerations}


\begin{frame}

  What is a Proof?
  
\end{frame}

\begin{frame}

  Defintions

  Syntctic Completeness
  Semantic Adequacy

\end{frame}

\begin{frame}[fragile]
% https://tikzcd.yichuanshen.de/#N4Igdg9gJgpgziAXAbVABwnAlgFyxMJZABgBoBmAXVJADcBDAGwFcYkQBlNGAYx2YBOAWwA6IgAQQAZmPEBxAGIgAvqXSZc+QijLFqdJq3YAZGAA8sPArI5YAXm1XrseAkXKk9NBizaJOAJ5gOPR8lrIAwhBCaIwwODBg8AhOIBguWu4U+j5G-hwwQvTB4RIAgrAAjsyhASpqaRqu2sgeAIw5hn4gKvowUADm8ESgUgLRSGQgOBBIbd5d7BwheHB4PEyyALLxABbQcAD89aPjQpM0M0geBr7sAErFIbIAEhAAKu+yAORtACwgGiMegAIxgjAACk1Mv4BFgBrscCcQGMJogpldEAAmBZ3fw8ZggyxMT7I1HndGXWaIP64vIgR7BeiyYwQAalcS-AFA0HgqEZNyw+GI3rKIA
\centering
\begin{tikzcd}
Lexicon\ Size                                                                                                                                          &  &  & Syntactic\ Completeness \\
                                                                                                                                                       &  &  & {}                      \\
                                                                                                                                                       &  &  &                         \\
Specturm\ of\ GF \arrow[uuu, "Statistical\ Methods?"] \arrow[rrr, "Ranta\ HoTT\ '14"'] \arrow[rrruuu, "cubicalTT"] \arrow[rrruu, "Ranta\ Logic\ '14"'] &  &  & Semantic\ Adequacy     
\end{tikzcd}

\end{frame}

\section{GF Grammars}


\begin{frame}

  

\end{frame}

% \begin{frame}



% \end{frame}



% \begin{frame}[fragile]
% \centering 

% \end{frame}


% \begin{frame}
%   \frametitle{Overview}

% \begin{enumerate}
% \item \textit{Negation} is a contentions notion, in a sense (but we'll
%   not get into this here)
% \item Buszkowski added axioms for a kind of negation in categorial
%   grammars
% \item Wansing presents a different kind, which leads to motivation
%   of connexive logic
% \end{enumerate}
% \end{frame}

% \begin{frame}
%   \frametitle{Negative Information}
%   To allow for the expression that ``'sleeps John' is an invalid
%   sentence'' (it's not just ``not valid'', it's \textit{invalid}, one could assign it the type ``$¬(s\backslash n)$''
  
%   There are many nice connections to algebra, and even category theory (Lambek calculus was inspired on that), but we won't be touching upon (no time).
  
% \end{frame}
  

% \section{Two Exercises}
% \subsection{Negative Normal Form}

% \begin{frame}


% \begin{alertblock}{Observation 3.4}
% For every type symbol $x$, $x'$ is in NNF and $\vdash_S x \Leftrightarrow x'$, for $S \in \{\textbf{NL}^\neg , \textbf{L}^\neg \}$.
% \end{alertblock}  
% \begin{columns}
% \frametitle{The Negation Normal Form}
% \begin{column}{0.4 \textwidth}
% \begin{exampleblock}{Definition 3.1: type symbol}
%   \begin{enumerate}
%     \item atomic type symbols $x,y,w,\dots$ are type symbols;
%     \item if $X$ and $Y$ are type symbols, also $(X * Y)$ is a type symbol, for $ * \in \{ \backslash , / , x \}$
%     \item if $X$ is a type symbol, also $\neg X$ is a type symbol $( X \neq Y \times Z)$;
%    % \item nothing else is a type symbol.
%   \end{enumerate}
% \end{exampleblock}
% \end{column}

%   \begin{column}{0.4 \textwidth}
%   \begin{exampleblock}{Negation Normal Form}
%   Define a function $'$ such that:
%   \begin{align*}
%     x' &= x \quad \text{\small (x atomic)}\\
%     (\neg x)' &= \neg x \quad \text{\small (x atomic)}\\
%     (\neg\neg X)' &= X' \\
%     (X * Y)' &= (X' * Y') \\
%     (\neg(Y/X))' &= ((\neg Y)'/X') \\
%     (\neg (X\backslash Y))' &= (X' \backslash (\neg Y)')
%   \end{align*}   
%     \end{exampleblock} 
%   \end{column}

% \end{columns}
  
% \end{frame}

% \begin{frame}

% % \begin{alertblock}{Observation 3.4}
% % For every type symbol $x$, $x'$ is in NNF and $\vdash_S x \Leftrightarrow x'$, for $S \in \{\textbf{NL}^\neg , \textbf{L}^\neg \}$.
% % \end{alertblock}  
% Proof: \begin{itemize}
%   \item[A.] $x$ is a type symbol $\rightarrow$ $x'$ is its negation normal form;
%   \item[B.] $x$ is a type symbol $\rightarrow$ $\vdash_{\small S} x \Leftrightarrow x'$
% \end{itemize}
% \pause
% \begin{block}{A.}
% Proof is straightforward by induction on the complexity of type symbols.\\
% %\alert{Intuitively:} negation is "pushed" inside in the cases $\backslash$ and $/$, removed in case $\neg \neg x$ \\

% \end{block}
% (note: $\neg (X * Y)$ is not a valid type symbol here).
  
% \end{frame}

% \begin{frame}
%   \begin{block}{B. ($\Rightarrow case$)}
% By induction on the complexity of $x$:
% \begin{enumerate}
%   \item <1-> $x$ is atomic: $x' = x$ and by $(id)$ $\vdash x \Rightarrow x'$;
%   \item <2-> $x= (y \times w)$: by IH  $\vdash y \Leftrightarrow y'$ and $\vdash w \Leftrightarrow w'$
%   \begin{prooftree}
%   \AxiomC{$y \Rightarrow y'$}
%   \AxiomC{$w \Rightarrow w'$}
%   \RightLabel{\tiny $(\to \times)$}
%   \BinaryInfC{$y, w \Rightarrow (y' \times x')$}
%   \RightLabel{\tiny $(\times \to)$}
%   \UnaryInfC{ $(y\times w) \Rightarrow (y' \times x')$}
%   \end{prooftree}
  
%   \item <2-> $x= (y / w)$: same IH
%    \begin{prooftree}
%   \AxiomC{$w' \Rightarrow w$}
%   \AxiomC{$y \Rightarrow y'$}
%   \RightLabel{\tiny $(/ \to)$}
%   \BinaryInfC{$(y/ w), w' \Rightarrow y'$}
%   \RightLabel{\tiny $(\to /)$}
%   \UnaryInfC{ $y/ w \Rightarrow y'/w'$}
%   \end{prooftree}
  
%   \item <2-> $x= (y \backslash w)$: dual.
% %    \begin{prooftree}
% %      \AxiomC{$y' \Rightarrow y$}
% %   \AxiomC{$w \Rightarrow w'$}
% %   \RightLabel{\tiny $(\backslash \to)$}
% %   \BinaryInfC{$y',(y\backslash w)\Rightarrow w'$}
% %   \RightLabel{\tiny $(\to \backslash)$}
% %   \UnaryInfC{ $y\backslash w \Rightarrow y'\backslash w'$}
% %   \end{prooftree}
% \end{enumerate}
% \end{block}

% \end{frame}

% \begin{frame}
%   \begin{block}{}
%   \begin{enumerate}
%     \item[5] $x = \neg y$: by IH $\vdash y \Leftrightarrow y' $  \\We can't deduce $\neg y \Leftrightarrow (\neg y)' $ directly. We need to look at $y$:
%     \begin{itemize}
%       \item <2-> $y$ atomic then $(\neg y)' = \neg y$ and $\vdash \neg y \Rightarrow \neg y$ by (id);
%       \item <2-> $y =\neg w$. We want a proof of $\neg \neg w \Leftrightarrow (\neg \neg w)'$. 
%       By IH we know that $\vdash w \Leftrightarrow w'$ 

%       \begin{prooftree}
%       \AxiomC{$w \Rightarrow w'$}
%     \RightLabel{\tiny $(\neg \neg \to)$}
%       \UnaryInfC{$ \neg  \neg w \Rightarrow w'$}
%      \RightLabel{\tiny $(\to \neg \neg )$}
%       \UnaryInfC{$ \neg  \neg w \Rightarrow \neg \neg w'$}
%       \end{prooftree} 
      
%       \item <3-> $y = w/z$. We want a proof of $\neg (w/z) \Rightarrow ((\neg w')/z')$ since $(\neg (w/z))' = ((\neg w')/z')$
%       By IH we can assume \[z \Leftrightarrow z' \qquad \neg w \Leftrightarrow (\neg w)'\]
 
%          \begin{prooftree}
%   \AxiomC{$z' \Rightarrow z$}
%   \AxiomC{$\neg w \Rightarrow (\neg w)'$}
%   \RightLabel{\tiny $(\neg / \to)$}
%   \BinaryInfC{$\neg (w/z), z' \Rightarrow (\neg w)'$}
%   \RightLabel{\tiny $(\to /)$}
%   \UnaryInfC{ $ \neg (w/z) \Rightarrow ((\neg w)'/z')$}
%   \end{prooftree}
%      \end{itemize}
%   \end{enumerate}
  
%   \end{block}
% \end{frame}

% \subsection{Proof in Agda}


% \begin{frame}
% \begin{columns}

% \begin{column}{0.4 \textwidth}
% \begin{exampleblock}{Definition 3.1: type symbol}
%   \begin{enumerate}
%     \item Atomic type symbols $x,y,w,\dots$ are type symbols;
%     \item if $X$ and $Y$ are type symbols, also $(X * Y)$ is a type symbol, for $ * \in \{ \backslash , / , x \}$
%     \item if $X$ is a type symbol, also $\neg X$ is a type symbol $( X \neq Y \times Z)$;
%     \item nothing else is a type symbol.
%   \end{enumerate}
% \end{exampleblock}
% \end{column}

%   \begin{column}{0.6 \textwidth}

%   % \begin{code}[hide]%
\>[0]\AgdaKeyword{module}\AgdaSpace{}%
\AgdaModule{Foo}\AgdaSpace{}%
\AgdaKeyword{where}\<%
\\
%
\\[\AgdaEmptyExtraSkip]%
\>[0]\AgdaKeyword{open}\AgdaSpace{}%
\AgdaKeyword{import}\AgdaSpace{}%
\AgdaModule{Data.List}\AgdaSpace{}%
\AgdaKeyword{using}\AgdaSpace{}%
\AgdaSymbol{(}\AgdaDatatype{List}\AgdaSymbol{;}\AgdaSpace{}%
\AgdaOperator{\AgdaFunction{\AgdaUnderscore{}++\AgdaUnderscore{}}}\AgdaSymbol{;}\AgdaSpace{}%
\AgdaOperator{\AgdaFunction{[\AgdaUnderscore{}]}}\AgdaSpace{}%
\AgdaSymbol{;}\AgdaSpace{}%
\AgdaInductiveConstructor{[]}\AgdaSymbol{;}%
\>[52]\AgdaOperator{\AgdaInductiveConstructor{\AgdaUnderscore{}∷\AgdaUnderscore{}}}\AgdaSpace{}%
\AgdaSymbol{)}\<%
\\
%
\\[\AgdaEmptyExtraSkip]%
\>[0]\AgdaKeyword{data}\AgdaSpace{}%
\AgdaDatatype{Nat}\AgdaSpace{}%
\AgdaSymbol{:}\AgdaSpace{}%
\AgdaPrimitive{Set}\AgdaSpace{}%
\AgdaKeyword{where}\<%
\\
\>[0][@{}l@{\AgdaIndent{0}}]%
\>[2]\AgdaInductiveConstructor{zero}\AgdaSpace{}%
\AgdaSymbol{:}\AgdaSpace{}%
\AgdaDatatype{Nat}\<%
\\
%
\>[2]\AgdaInductiveConstructor{suc}\AgdaSpace{}%
\AgdaSymbol{:}\AgdaSpace{}%
\AgdaDatatype{Nat}\AgdaSpace{}%
\AgdaSymbol{→}\AgdaSpace{}%
\AgdaDatatype{Nat}\<%
\\
\>[0]\<%
\end{code}

\begin{code}%
\>[0]\<%
\\
%
\\[\AgdaEmptyExtraSkip]%
\>[0]\AgdaKeyword{data}\AgdaSpace{}%
\AgdaDatatype{tSymb}\AgdaSpace{}%
\AgdaSymbol{:}\AgdaSpace{}%
\AgdaPrimitive{Set}\AgdaSpace{}%
\AgdaKeyword{where}\<%
\\
\>[0][@{}l@{\AgdaIndent{0}}]%
\>[2]\AgdaInductiveConstructor{base}\AgdaSpace{}%
\AgdaSymbol{:}\AgdaSpace{}%
\AgdaDatatype{Nat}\AgdaSpace{}%
\AgdaSymbol{→}\AgdaSpace{}%
\AgdaDatatype{tSymb}\<%
\\
%
\>[2]\AgdaInductiveConstructor{\textasciitilde{}}\AgdaSpace{}%
\AgdaSymbol{:}\AgdaSpace{}%
\AgdaDatatype{tSymb}\AgdaSpace{}%
\AgdaSymbol{→}\AgdaSpace{}%
\AgdaDatatype{tSymb}\<%
\\
%
\>[2]\AgdaOperator{\AgdaInductiveConstructor{\AgdaUnderscore{}\textbackslash{}\textbackslash{}\AgdaUnderscore{}}}\AgdaSpace{}%
\AgdaSymbol{:}\AgdaSpace{}%
\AgdaDatatype{tSymb}\AgdaSpace{}%
\AgdaSymbol{→}\AgdaSpace{}%
\AgdaDatatype{tSymb}\AgdaSpace{}%
\AgdaSymbol{→}\AgdaSpace{}%
\AgdaDatatype{tSymb}\<%
\\
%
\>[2]\AgdaOperator{\AgdaInductiveConstructor{\AgdaUnderscore{}//\AgdaUnderscore{}}}\AgdaSpace{}%
\AgdaSymbol{:}\AgdaSpace{}%
\AgdaDatatype{tSymb}\AgdaSpace{}%
\AgdaSymbol{→}\AgdaSpace{}%
\AgdaDatatype{tSymb}\AgdaSpace{}%
\AgdaSymbol{→}\AgdaSpace{}%
\AgdaDatatype{tSymb}\<%
\\
\>[0]\<%
\end{code}


%   \end{column}

% \end{columns}
% \end{frame}


% \begin{frame}


% \begin{columns}
% \begin{column}{0.4 \textwidth}
% \begin{exampleblock}{Definition 3.2: categorial entailment}

%   \begin{prooftree}
%   \AxiomC{}
%   \RightLabel{\tiny $(id)$}
%   \UnaryInfC{$x \Rightarrow x$}
%   \end{prooftree}

% 	\begin{prooftree}
%   \AxiomC{$x,X \Rightarrow y$}
%   \RightLabel{\tiny $(\to \backslash )$}
%   \UnaryInfC{$X \Rightarrow (x\backslash y)$}
%   \end{prooftree}
  
%   \begin{prooftree}
%   \AxiomC{$X \Rightarrow x$}
%   \AxiomC{$Y,y,Y' \Rightarrow z$}
%   \RightLabel{\tiny $(\backslash \to)$}
%   \BinaryInfC{$Y,X,(x\backslash y),Y' \Rightarrow z$}
%   \end{prooftree}
  

% \end{exampleblock}
% \end{column}


%   \begin{column}{0.7 \textwidth}

%   % \begin{code}[hide]%
\>[0]\AgdaKeyword{module}\AgdaSpace{}%
\AgdaModule{Foo1}\AgdaSpace{}%
\AgdaKeyword{where}\<%
\\
%
\\[\AgdaEmptyExtraSkip]%
\>[0]\AgdaComment{-- open import Data.List using (List; \AgdaUnderscore{}++\AgdaUnderscore{}; [\AgdaUnderscore{}] ; [];  \AgdaUnderscore{}∷\AgdaUnderscore{} ) renaming (\AgdaUnderscore{}∷\AgdaUnderscore{} to \AgdaUnderscore{},\AgdaUnderscore{})}\<%
\\
\>[0]\AgdaKeyword{open}\AgdaSpace{}%
\AgdaKeyword{import}\AgdaSpace{}%
\AgdaModule{Data.List}\AgdaSpace{}%
\AgdaKeyword{renaming}\AgdaSpace{}%
\AgdaSymbol{(}\AgdaOperator{\AgdaInductiveConstructor{\AgdaUnderscore{}∷\AgdaUnderscore{}}}\AgdaSpace{}%
\AgdaSymbol{to}\AgdaSpace{}%
\AgdaOperator{\AgdaInductiveConstructor{\AgdaUnderscore{},\AgdaUnderscore{}}}\AgdaSymbol{)}\<%
\\
%
\\[\AgdaEmptyExtraSkip]%
\>[0]\AgdaKeyword{data}\AgdaSpace{}%
\AgdaDatatype{Nat}\AgdaSpace{}%
\AgdaSymbol{:}\AgdaSpace{}%
\AgdaPrimitive{Set}\AgdaSpace{}%
\AgdaKeyword{where}\<%
\\
\>[0][@{}l@{\AgdaIndent{0}}]%
\>[2]\AgdaInductiveConstructor{zero}\AgdaSpace{}%
\AgdaSymbol{:}\AgdaSpace{}%
\AgdaDatatype{Nat}\<%
\\
%
\>[2]\AgdaInductiveConstructor{suc}\AgdaSpace{}%
\AgdaSymbol{:}\AgdaSpace{}%
\AgdaDatatype{Nat}\AgdaSpace{}%
\AgdaSymbol{→}\AgdaSpace{}%
\AgdaDatatype{Nat}\<%
\\
%
\\[\AgdaEmptyExtraSkip]%
\>[0]\AgdaKeyword{data}\AgdaSpace{}%
\AgdaDatatype{tSymb}\AgdaSpace{}%
\AgdaSymbol{:}\AgdaSpace{}%
\AgdaPrimitive{Set}\AgdaSpace{}%
\AgdaKeyword{where}\<%
\\
\>[0][@{}l@{\AgdaIndent{0}}]%
\>[2]\AgdaInductiveConstructor{base}\AgdaSpace{}%
\AgdaSymbol{:}\AgdaSpace{}%
\AgdaDatatype{Nat}\AgdaSpace{}%
\AgdaSymbol{→}\AgdaSpace{}%
\AgdaDatatype{tSymb}\<%
\\
%
\>[2]\AgdaInductiveConstructor{\textasciitilde{}}\AgdaSpace{}%
\AgdaSymbol{:}\AgdaSpace{}%
\AgdaDatatype{tSymb}\AgdaSpace{}%
\AgdaSymbol{→}\AgdaSpace{}%
\AgdaDatatype{tSymb}\<%
\\
%
\>[2]\AgdaOperator{\AgdaInductiveConstructor{\AgdaUnderscore{}\textbackslash{}\textbackslash{}\AgdaUnderscore{}}}\AgdaSpace{}%
\AgdaSymbol{:}\AgdaSpace{}%
\AgdaDatatype{tSymb}\AgdaSpace{}%
\AgdaSymbol{→}\AgdaSpace{}%
\AgdaDatatype{tSymb}\AgdaSpace{}%
\AgdaSymbol{→}\AgdaSpace{}%
\AgdaDatatype{tSymb}\<%
\\
%
\>[2]\AgdaOperator{\AgdaInductiveConstructor{\AgdaUnderscore{}//\AgdaUnderscore{}}}\AgdaSpace{}%
\AgdaSymbol{:}\AgdaSpace{}%
\AgdaDatatype{tSymb}\AgdaSpace{}%
\AgdaSymbol{→}\AgdaSpace{}%
\AgdaDatatype{tSymb}\AgdaSpace{}%
\AgdaSymbol{→}\AgdaSpace{}%
\AgdaDatatype{tSymb}\<%
\end{code}

\begin{code}%
\>[0]\<%
\\
\>[0]\AgdaFunction{Ctx}\AgdaSpace{}%
\AgdaSymbol{:}\AgdaSpace{}%
\AgdaPrimitive{Set}\<%
\\
\>[0]\AgdaFunction{Ctx}\AgdaSpace{}%
\AgdaSymbol{=}\AgdaSpace{}%
\AgdaDatatype{List}\AgdaSpace{}%
\AgdaDatatype{tSymb}\<%
\\
%
\\[\AgdaEmptyExtraSkip]%
\>[0]\AgdaKeyword{data}\AgdaSpace{}%
\AgdaOperator{\AgdaDatatype{\AgdaUnderscore{}=>\AgdaUnderscore{}}}\AgdaSpace{}%
\AgdaSymbol{:}\AgdaSpace{}%
\AgdaFunction{Ctx}\AgdaSpace{}%
\AgdaSymbol{→}\AgdaSpace{}%
\AgdaDatatype{tSymb}\AgdaSpace{}%
\AgdaSymbol{→}\AgdaSpace{}%
\AgdaPrimitive{Set}\AgdaSpace{}%
\AgdaKeyword{where}\<%
\\
\>[0][@{}l@{\AgdaIndent{0}}]%
\>[2]\AgdaInductiveConstructor{id-axiom}\AgdaSpace{}%
\AgdaSymbol{:}\AgdaSpace{}%
\AgdaSymbol{(}\AgdaBound{x}\AgdaSpace{}%
\AgdaSymbol{:}\AgdaSpace{}%
\AgdaDatatype{tSymb}\AgdaSymbol{)}\AgdaSpace{}%
\AgdaSymbol{→}\AgdaSpace{}%
\AgdaOperator{\AgdaFunction{[}}\AgdaSpace{}%
\AgdaBound{x}\AgdaSpace{}%
\AgdaOperator{\AgdaFunction{]}}\AgdaSpace{}%
\AgdaOperator{\AgdaDatatype{=>}}\AgdaSpace{}%
\AgdaBound{x}\<%
\\
%
\>[2]\AgdaInductiveConstructor{\textbackslash{}\textbackslash{}r}\AgdaSpace{}%
\AgdaSymbol{:}%
\>[66I]\AgdaSymbol{(}\AgdaBound{Γ}\AgdaSpace{}%
\AgdaSymbol{:}\AgdaSpace{}%
\AgdaFunction{Ctx}\AgdaSpace{}%
\AgdaSymbol{)(}\AgdaBound{x}\AgdaSpace{}%
\AgdaBound{y}\AgdaSpace{}%
\AgdaSymbol{:}\AgdaSpace{}%
\AgdaDatatype{tSymb}\AgdaSymbol{)}\<%
\\
\>[.][@{}l@{}]\<[66I]%
\>[8]\AgdaSymbol{→}\AgdaSpace{}%
\AgdaSymbol{(}\AgdaSpace{}%
\AgdaBound{x}\AgdaSpace{}%
\AgdaOperator{\AgdaInductiveConstructor{,}}\AgdaSpace{}%
\AgdaBound{Γ}\AgdaSymbol{)}\AgdaSpace{}%
\AgdaOperator{\AgdaDatatype{=>}}\AgdaSpace{}%
\AgdaBound{y}\<%
\\
%
\>[8]\AgdaSymbol{→}\AgdaSpace{}%
\AgdaBound{Γ}\AgdaSpace{}%
\AgdaOperator{\AgdaDatatype{=>}}\AgdaSpace{}%
\AgdaSymbol{(}\AgdaSpace{}%
\AgdaBound{x}\AgdaSpace{}%
\AgdaOperator{\AgdaInductiveConstructor{\textbackslash{}\textbackslash{}}}\AgdaSpace{}%
\AgdaBound{y}\AgdaSpace{}%
\AgdaSymbol{)}\<%
\\
%
\>[2]\AgdaInductiveConstructor{\textbackslash{}\textbackslash{}l}\AgdaSpace{}%
\AgdaSymbol{:}%
\>[87I]\AgdaSymbol{(}\AgdaBound{Δ}\AgdaSpace{}%
\AgdaBound{Δ'}\AgdaSpace{}%
\AgdaBound{Γ}\AgdaSpace{}%
\AgdaSymbol{:}\AgdaSpace{}%
\AgdaFunction{Ctx}\AgdaSpace{}%
\AgdaSymbol{)(}\AgdaBound{x}\AgdaSpace{}%
\AgdaBound{y}\AgdaSpace{}%
\AgdaBound{z}\AgdaSpace{}%
\AgdaSymbol{:}\AgdaSpace{}%
\AgdaDatatype{tSymb}\AgdaSymbol{)}\<%
\\
\>[.][@{}l@{}]\<[87I]%
\>[8]\AgdaSymbol{→}\AgdaSpace{}%
\AgdaSymbol{(}\AgdaBound{Δ}\AgdaSpace{}%
\AgdaOperator{\AgdaFunction{++}}\AgdaSpace{}%
\AgdaOperator{\AgdaFunction{[}}\AgdaSpace{}%
\AgdaBound{y}\AgdaSpace{}%
\AgdaOperator{\AgdaFunction{]}}\AgdaSpace{}%
\AgdaOperator{\AgdaFunction{++}}\AgdaSpace{}%
\AgdaBound{Δ'}\AgdaSymbol{)}\AgdaSpace{}%
\AgdaOperator{\AgdaDatatype{=>}}\AgdaSpace{}%
\AgdaBound{z}\<%
\\
%
\>[8]\AgdaSymbol{→}\AgdaSpace{}%
\AgdaBound{Γ}\AgdaSpace{}%
\AgdaOperator{\AgdaDatatype{=>}}\AgdaSpace{}%
\AgdaBound{x}\<%
\\
%
\>[8]\AgdaComment{-----------------------------}\<%
\\
%
\>[8]\AgdaSymbol{→}%
\>[109I]\AgdaSymbol{(}\AgdaBound{Δ}\AgdaSpace{}%
\AgdaOperator{\AgdaFunction{++}}\AgdaSpace{}%
\AgdaBound{Γ}\AgdaSpace{}%
\AgdaOperator{\AgdaFunction{++}}\AgdaSpace{}%
\AgdaOperator{\AgdaFunction{[}}\AgdaSpace{}%
\AgdaBound{x}\AgdaSpace{}%
\AgdaOperator{\AgdaInductiveConstructor{\textbackslash{}\textbackslash{}}}\AgdaSpace{}%
\AgdaBound{y}\AgdaSpace{}%
\AgdaOperator{\AgdaFunction{]}}\AgdaSpace{}%
\AgdaOperator{\AgdaFunction{++}}\AgdaSpace{}%
\AgdaBound{Δ'}\AgdaSymbol{)}\<%
\\
\>[.][@{}l@{}]\<[109I]%
\>[10]\AgdaOperator{\AgdaDatatype{=>}}\AgdaSpace{}%
\AgdaBound{z}\<%
\\
\>[0]\<%
\end{code}


%   \end{column}

% \end{columns}
% \end{frame}




\end{document}
